\chapter*{Introduction}
\addcontentsline{toc}{chapter}{Introduction}

\paragraph{}
Learning management systems are becoming ubiquitous technology adopted at institutions of higher learning. Before these systems can be considered effective the user experience must be studied and analyzed to provide the optimum solution to meet pedagogical needs of both faculty and students. \cite{machado2007blackboard}

\paragraph{}
There are already many open-source learning management system but none of them matched our requirement to be easily extendable, easy to use and lightweight. For example Moodle, is too complicated to extend and does not allow us to do rapid development in new scenarios.

\paragraph{}
As a result of this, during the year 2014 and 2015, the learning management system Courses 2 was developed by a student, Jakub Culik under supervision of 3 teachers and multiple developers. This system is fully modular, with custom modules for assignments, notes, quises or results. Until now, this system has been used for 4 terms with about 250 users registered and X courses taught. During the usage of this system, some problems that needed to be solved arose.
