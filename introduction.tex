\chapter*{Introduction}
\addcontentsline{toc}{chapter}{Introduction}

\paragraph{}
Learning management systems are becoming ubiquitous technology adopted at institutions of higher learning. Before these systems can be considered effective the user experience must be studied and analyzed to provide the optimum solution to meet pedagogical needs of both faculty and students. \cite{machado2007blackboard}

\paragraph{}
There are already many open-source learning management system but none of them matched our requirement to be easily extendable, easy to use and lightweight. For example Moodle, is too complicated to extend and does not allow us to do rapid development in new scenarios.

\paragraph{}
As a result of this, during the year 2014 and 2015, the learning management system Courses 2 was developed by a student, Jakub Culik under supervision of 3 teachers and multiple developers. This system is fully modular, with custom modules for assignments, notes, quises or results. Until now, this system has been used for 4 terms with about 250 users registered and X courses taught. During the usage of this system, some problems that needed to be solved arose.

\section*{Problems, solution and contribution}

\paragraph{}
Development of this system was finished in the spring of 2015. Since then, various bugs were discovered that needed to be solved. One of the purposes of this work is to provide support for this system and fix these issues.

\paragraph{}
During usage of Courses 2 we descovered that it is important to enable peer review and improved submissions in assignments module. The flow is following: student is assigned an assignment and submits his solution. After that, this solution is peer reviewed and student gets feedback from other students. Then, he is enabled to submit an improved solution of this assignemnt. This work was also published as article "Peer Review Support in a Virtual Learning Environment"  \cite{homola2016peer}.

\paragraph{}
Another extension to Courses 2 system was support for team assignment management. Team assignemnts has appeared to be important in the process of learing. This extension was first used by "Modern approaches to webdesign" course and used by about 50 students. It was also aim of one diploma thesis to develop this extension.

\paragraph{}
Hovewer the most important part of this work was to implement easy to use web mirroring. In some courses, students were submitting URLs of their web projects, not the sources itself. This enabled the students to use technologies which they wanted to use to develop their web pages and teacher reviewed only the rendered page. The problem was that student was able to alter his project anytime he wanted to do, so there were plenty of room for cheating. We needed to develop the technology to auto backup the rendered pages in such cases. In this work, we present simple and elegant solution of this problem.
